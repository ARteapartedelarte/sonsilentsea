\section{Conclusiones} 

\subsection{Resultado del experimento}
\begin{frame}{Resultado}
\begin{columns}
  \begin{column}{0.5\textwidth}
    \textbf{\underline{Blender}}
    
    \textbf{2745 líneas de código ``comparables''}

	\begin{figure}
		\includegraphics[scale=0.15]{python_logo} 
	\end{figure}
  \end{column}

  \begin{column}{0.5\textwidth}
    \textbf{\underline{OGRE}}
    
    \textbf{15050 líneas de código ``comparables''}

	\begin{figure}
		\includegraphics[scale=0.8]{C_plus_plus} 
	\end{figure}
  \end{column}
\end{columns}

\vspace{0.5cm}
Según la estimación (en la que he tratado de restringir al máximo los
efectos al cambio de lenguaje de programación) el código en C++ parece
requerir 5.5 veces más líneas de código que una implementación en Python.

\vspace{0.5cm}
La medida queda fuera del valor que obtuvimos muestreando
algunos algoritmos de ordenación ($2.7 \pm 1.6$).

\end{frame}

\subsection{Conclusiones}
\begin{frame}{Conclusiones}
\begin{itemize}
	\item Existe una gran cantidad de información y comparaciones fiables sobre la
	velocidad de códigos implementados en ambos lenguajes. \pause
	\item Existen estimaciones del número de errores que se cometen en cada
	lenguaje por línea de código escrita (0.69 vs. 0.005 bugs per KLOC). \pause
	\item No parecen existir estudios serios acerca de la diferencia de esfuerzo
	en el desarrollo dependiendo del lenguaje empleado.
	Por ejemplo, si en \CC $\,$ se requiren 10 líneas más de código que en Python
	para llevar a cabo la misma tarea, el índice de errores estará artificialmente
	incrementado un orden de magnitud. \pause
	\item Se ha tratado de estimar la influencia del lenguaje de programación
	en una aplicación compleja, usando como indicador el número de líneas
	de código. Para ello se ha tratado de eliminar todo elemento ``no comparable''.
	\pause
	\item Ambas aplicaciones son demasiado diferentes, así que el resultado no
	debe considerarse una buena medida. \pause
	\item Tal y como se podía esperar en \CC $\,$ se ha obtenido un incremento
	significativo de las líneas de código necesarias. \pause
	\item Tampoco se ha abordado la dependencia con el ámbito de aplicación.
\end{itemize}
\end{frame}

\section{Preguntas}
\begin{frame}{Preguntas}
\begin{center}
{\Huge ¿Preguntas?}
\end{center}
\end{frame}
