\subsection{Física}

\begin{frame}{Creando la física}
Partimos de:
\begin{enumerate}
	\item Un objeto poco detallado para manejar la física.
	\item Dos objetos detallados sin propiedades físicas. Estos objetos tan sólo
	estarán ligados al objeto físico  que será invisible.
\end{enumerate}

Acciones a tomar:
\begin{enumerate}
	\item Retiramos la física a los dos objetos ``visuales''.
	\item Establecemos el controlador físico como dinámico:
	\begin{enumerate}
		\item $m = 4.6 \mathrm{ton}$
		\item $r = 1.3 \mathrm{m}$
		\item $r_{factor} = 1$
	\end{enumerate}
	También eliminamos todos los efectos disipativos.
\end{enumerate}

\end{frame}

\begin{frame}{Física implementada ``out of the box''}
El objeto ahora tiene la física implementada ``de serie'', y por tanto si
empezamos la simulación el objeto caerá irremediablemente.

Queremos por tanto implementar los efectos del agua.
\end{frame}
